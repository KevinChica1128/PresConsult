\documentclass[12pt]{beamer}
\usetheme{CambridgeUS}
\usepackage[utf8]{inputenc}
\usepackage[spanish]{babel}
\usepackage{amsmath}
\usepackage{amsfonts}
\usepackage{amssymb}
\usepackage{graphicx}
\usepackage{ragged2e}
\author{Kevin Garcia - Alejandro Vargas - Alejandro Soto}
\title{Consultoría}
%\setbeamercovered{transparent} 
%\setbeamertemplate{navigation symbols}{} 
%\logo{} 
%\institute{} 
%\date{} 
%\subject{} 
\begin{document}

\justifying
\begin{frame}
\titlepage
\end{frame}

%\begin{frame}
%\tableofcontents
%\end{frame}

\begin{frame}
\frametitle{Problema y contacto}
\begin{block}{Problema}
\justifying
Se quiere evaluar el efecto de un nuevo método de enseñanza, denominado flipped classroom en un grupo de estudiantes de la escuela nacional del deporte. 
\end{block}
\begin{block}{Contacto}
\begin{itemize}
\item Wilson Canizales: Docente y director técnico de proyección social en la escuela nacional del deporte.
\end{itemize}
\end{block}
\end{frame}

\begin{frame}
\frametitle{Preguntas}
\begin{itemize}
\item[1.] ¿En qué fase se encuentra el proyecto?
\item[2.] ¿Cuál es la población de estudio?
\item[3.] ¿De qué se trata el nuevo método de enseñanza?¿Cuáles son sus diferencias con el método tradicional?
\item[4.] ¿Ya se hizo la recolección de datos?
\item[5.] ¿Qué variables se midieron?
\item[6.] ¿Qué instrumentos de medición se utilizaron? ¿Están validados?
\item[7.] ¿Cómo se seleccionaron los estudiantes para el grupo de control y grupo intervenido?
\item[8.] ¿Cómo se definió el tamaño de muestra?
\item[9.] ¿Qué control se tuvo aplicando los instrumentos de medición?
\end{itemize}
\end{frame}

\begin{frame}
\frametitle{Variables}
\scalebox{0.45}[0.8]{
\begin{tabular}{|c|c|c|c|}
\hline 
Variable & Definición operacional & Tipo de variable & Escala de medida \\ 
\hline 
Nombre & Designación verbal que se le da a una persona & Cualitativa & Nominal \\ 
Sexo & Condición orgánica, masculina o femenina, de los animales y las plantas & Cualitativa & Nominal \\ 
Edad & Tiempo que ha vivido una persona desde su nacimiento & Cuantitativa Continua & Intervalos \\ 
Estado civil & Situación de las personas físicas determinada por sus relaciones de familia & Cualitativa & Ordinal \\ 
Número de hijos & Cantidad de hijos que posee & Cuantitativa Discreta & Razón \\ 
Estrato Socioecónomico & Clase o grupo en el que está la persona de acuerdo a su poder adquisitivo & Cualitativa & Ordinal \\ 
Estilo de aprendizaje & Método o estrategia preferente para aprender & Cualitativa & Nominal \\ 
Estilo de uso del espacio virtual & Estilo o forma en la que se usa la web & Cualitativa & Nominal \\ 
Experiencia de prendizaje & Percepción del estudiante respecto a su experiencia de aprendizaje  & Cuantitativa Continua & Intervalos \\ 
Ambiente de clase & Percepción del estudiante respecto al ambiente de aula & Cuantitativa Continua & Intervalos \\ 
Percepción FC & Percepción del estudiante frente a la metodología flipedd classroom & Cuantitativa Continua & Intervalos \\
Nota & Nota final obtenida en el curso & Cuantitativa continua & Intervalos \\ 
\hline 
\end{tabular} 
}
\end{frame}

\begin{frame}
\frametitle{Plan de análisis}
Inicialmente se realizará un análisis descriptivo, donde los datos serán resumidos en medidas de tendencia central (media, mediana, moda), y de dispersión (desviación estándar, cuartiles) para las variables cuantitativas de acuerdo a su distribución. Las variables cualitativas se describirán mediante frecuencias absolutas y relativas. La validación del supuesto de normalidad se realizará mediante la prueba de Shapiro-Wilk.

~\\Posteriormente, en el análisis bivariado se realizará la comparación de las variables entre los grupos por medio de las pruebas de Chi-Cuadrado o Fisher, e intra grupo por medio del test de McNemar o Q de Cochran para las variables cualitativas. Para las variables cuantitativas, las comparaciones se realizarán por medio de la prueba t-student, en el caso que no se cumpla la normalidad se utilizará test no paramétricos como el test de Wilcoxon y U de Mann Whitney.
\end{frame}

\begin{frame}
\frametitle{Costeo}
\begin{columns}[T] % align columns
\begin{column}{.48\textwidth}
\begin{table}[htbp]
\scalebox{0.8}[0.8]{
    \begin{tabular}{|lr|}
    \hline
    \multicolumn{2}{|c|}{\textbf{Costo}} \\
    \hline
    \textbf{Métodos estadísticos} &  \\
    Depuracion de base de datos & 210000 \\
    Análisis descriptivo & 250000 \\
    Análisis inferencial & 400000 \\
    Pruebas de Hipótesis & 300000 \\
    Procesamiento de resultados & 90000 \\
    \textbf{Recursos para operación} &  \\
    Uso de software & 100000 \\
    Papeleria & 70000 \\
    Servicios de información & 200000 \\
    Viáticos (transporte) & 180000 \\
    \textbf{TOTAL} & 1800000 \\
    \hline
    \end{tabular}%
    }
  \label{tab:addlabel}%
\end{table}%
\end{column}%
\hfill%
\begin{column}{.48\textwidth}
% Table generated by Excel2LaTeX from sheet 'Hoja1'
\begin{table}[htbp]
\scalebox{0.7}[0.8]{
    \begin{tabular}{|lc|}
    \hline
    \multicolumn{2}{|c|}{\textbf{Costo}} \\
    \hline
    \textbf{Precio/hora (25000)} &  \\
    Horas semanales & 8 horas \\
    Tiempo de entrega del proyecto & 1 meses \\
    Horas  trabajadas por persona & 32 horas \\
    Total de horas  & 96 horas \\
    \textbf{TOTAL} & 2400000 \\
    \hline
    \end{tabular}%
    }
  \label{tab:addlabel}%
\end{table}%

\end{column}%
\end{columns}
% Table generated by Excel2LaTeX from sheet 'Hoja1'


\end{frame}


\end{document}
